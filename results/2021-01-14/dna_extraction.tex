\documentclass[a4paper]{article}
\usepackage[utf8]{inputenc}
\usepackage[T1]{fontenc}
\author{J. Ignacio Lucas Lledó}
\title{DNA extraction from fish fin preserved in ethanol}
\begin{document}
\section{Material}
\begin{itemize}
\item DNeasy Blood \& Tissue Kit 250 (Qiagen).
\item RNase A.
\item Liquid nitrogen.
\item Electric screw driver (recommended).
\item Autoclaved sand.
\item Clean plastic pestles.
\item Ice.
\item Parafilm.
\item Ethanol.
\item Vortex, spinner, pipettes, tips, 1.5 ml tubes.
\item Rocking incubation platform.
\item Centrifuge.
\end{itemize}

\section{First session}
\begin{enumerate}
\item Buffer ATL may form precipitate upon storage. If necessary, warm to 56$^\circ$C until precipitates fully dissolve.
\item Cut a small piece of tissue, about 8 mm$^3$ ($2\times 2\times 2$ mm cube), and put it in a 1.5 ml tube and immediately, on ice. Add a small amount of autoclaved sea sand. Keep the tube open and deep the bottom of it in the liquid nitrogen to freeze the sample, with care not to freeze your fingertips. Use the screw driver with a pestle fitted on it to grind the tissue. Press hard the pestle against the tube. Alternate between grinding and cooling the tube down in the liquid nitrogen.
\item When the tissue is powdered, let the tube recover a temperature >0$^\circ$C. Add 180 $\mu$l of buffer ATL and remove the pestle with care of leaving the liquid in the tube. Then, add 20 $\mu$l of proteinase K, vortex, and incubate at 56$^\circ$C overnight on a rocking platform. Make sure to seal the tubes with parafilm to reduce evaporation.
\end{enumerate}

\section{Second session}
\begin{enumerate}
\item Just like the ATL buffer, buffer AL may also form precipitates. Warm it up if necessary to dissolve them. Make sure that buffers AW1 and AW2 have the ethanol added.
\item After the incubation, bring samples to room temperature. Centrifuge for 2 min at 8000 $\times g$ (10000 rpm?), at 20$^\circ$C. Transfer supernatant to a new tube. This removes the sand.
\item Add 4 $\mu$l RNase A (100 mg/ml), mix by vortexing, and incubate for 2 min at room temperature.
\item Vortex for 15 s Add 200 $\mu$l buffer AL to the sample, and mix thoroughly by vortexing. Then add 200 $\mu$l ethanol (96-100\%), and mix again thoroughly by vortexing. It is essential that the sample, buffer AL, and ethanol are mixed immediately and thoroughly by vortexing or pipetting to yield a homogeneous solution. Buffer AL and ethanol can be premixed and added together in one step to save time when processing multiple samples.
\item Pipet the mixture from the previous step (including any precipitate) into the DNeasy Mini spin column placed in a 2 ml collection tube. Centrifuge at $\geq6000 \times g$ (8000 rpm) for 1 min. Discard flow-through and collection tube. 
\item Place the DNeasy Mini spin column in a new 2 ml collection tube, add 500 $\mu$lk buffer AW1, and centrifuge for 1 min at $\geq6000 \times g$ (8000 rpm). Discard flow-through and collection tube.
\item Place the DNeasy Mini spin column in a new 2 ml collection tube, add 500 $\mu$l buffer AW2, and centrifuge for 3 min at 20000$\geq \times g$ (14000 rpm) to dry the DNeasy membrane. Discard flow-through and collection tube. It is important to dry the membrane, since residual ethanol may interfere with subsequent reactions. This centrifugation step ensures that no residual ethanol will be carried over during the following elution. Following the centrifugation step, remove the DNeasy Mini spin column carefully so that the column does not come into contact with the flow-through, since this will result in carryover ethanol. If carryover of ethanol occurs, empty the collection tube, then reuse it in another centrifugation for 1 min at 20000$\times g$ (14000 rpm).
\item Place the DNeasy Mini spin column in a clean 1.5 ml or 2 ml microcentrifuge tube (not provided in the DNeasy kit), and pipet 200 $\mu$l buffer AE directly onto the DNeasy membrane. Incubate at room temperature for 1 min, and then centrifuge for 1 min at $\geq6000 \times g$ (8000 rpm) to elute. Elution with 100 $\mu$l (instead of 200 $\mu$l) increases the final DNA concentration in the eluate, but also decreases the overall DNA yield. For maximum DNA yield, repeat elution once more. Do not elute more than 200 $\mu$l into a 1.5 ml microcentrifuge tube because the DNeasy Mini spin column will come into contact with the eluate.
\end{enumerate}
\end{document}